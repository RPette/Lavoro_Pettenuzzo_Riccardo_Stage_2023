\documentclass[a4paper, notitlepage, 12pt]{article}
    \title{\textcolor{black}{\emph{PCTO BRETON SPA}}}
    \author{PETTENUZZO RICCARDO}%\thanks{Ringrazio Federico Milan, Luca Malorzo, De Zuani Stefano e l'intero DIGI Hub per l'esperienza}
    \date{}
    \pagestyle{empty}
    \pagenumbering{Roman}
    

\usepackage[utf8]{inputenc, xcolor}
\usepackage[paperwidth=597pt, paperheight=815pt, textheight = 721pt, voffset = 0pt]{geometry}
\setlength{\footnotesep}{10.75pt}
\setlength{\skip\footins}{45pt}

\begin{document}
    \normalsize
    \maketitle
    \begin{abstract}
        \noindent \rule{\linewidth}{.6pt}\par
        \noindent L'azienda Breton necessita che il processo di controllo dell'usura del disco dentellato usato nella
        macchina per il taglio delle lastre, venga automatizzato grazie ad un programma sviluppato in Python.
        Il programma elabora la foto dello sfrido scattata da una fotocamera montanta sulla macchina, viene calcolata
        la differenza di spessore lungo lo sfrido e quando quest'ultima oltrepassa un valore nell'ordine dei centesimi 
        di millimetro, l'utensile necessita di essere sostituito.
        \par
        \noindent \rule{\linewidth}{.6pt}  
    \end{abstract}
    \section{Obiettivo e Descrizione  del Progetto}\par
        \noindent L'azienda \textbf{Breton} dispone di macchine impiegate nel taglio di lastre di vario materiale,
        una di essere utilizza un \emph{mandrino}\footnote[0]{\indent Attrezzo della macchina operatrice con funzione di sostegno per l'utensile, o di supporto per il pezzo durante la lavorazione.} che sostiene un disco dentellato.
        La lastra viene posizionata nel piano della macchina, sorretta da delle ventose che ne bloccano il movimento,
        quando la macchina viene azionata parte il programma di taglio inserito, il disco scende e grazie ad una
        soluzione acquosa la lastra viene tagliata. Il disco è soggetto ad \textbf{usura}, che viene periodicamente controllata da un'addetto esperto e la 
        sostituzione dell'utensile avviene quando la larghezza dello \emph{sfrido}\footnote[1]{\indent Lo sfrido è lo scarto di lavorazione delle piastrelle che si crea durante la fase di posa.} differisce
        nella lunghezza di quest'ultimo. Per permettere all'addetto di controllare lo stato dello strumento la lastra
        non viene tagliata completamente nella sua profondità.\par
        \noindent L'obiettivo del progetto è quello di \textbf{automatizzare} il processo di controllo dello stato di usura 
        del disco, quindi realizzare un programma in Python\footnote[2]{\indent Python è un linguaggio di programmazione utilizzato nelle applicazionie web, sviluppo software, data science e machine learning} che elabora una foto 
        scattata da una fotocamera montata sulla macchina, in modo da ottenere la \emph{differenza di spessore} lungo lo sfrido, 
        il manutentore interviene nel caso in cui la differenza di spessore sia sopra un certo valore in centesimi di millimetro.\par
        \noindent Il programma utilizza la libreria \textbf{OpenCV}\footnote[3]{\indent OpenCV è una potente libreria che consente di eseguire operazioni di elaborazione delle immagini, deep learning, machine learning e computer vision su feed video in diretta.} per lo scatto della foto, successivamente 
        per la sua elaborazione e infine per la visione finale dei processo eseguiti per il ricavo del risultato.
    \pagebreak
    \begin{abstract}
        \noindent \rule{\linewidth}{.6pt}\par

        \noindent \rule{\linewidth}{.6pt} 
    \end{abstract}    
    \section{Primo metodo di Soluzione del problema}
\end{document}
